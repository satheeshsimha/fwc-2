\documentclass[12pt]{article}
\usepackage{graphicx}
\usepackage[none]{hyphenat}
\usepackage{graphicx}
\usepackage{listings}
\usepackage[english]{babel}
\usepackage{graphicx}
\usepackage{caption} 
\usepackage{booktabs}
\usepackage{array}
\usepackage{amssymb} % for \because
\usepackage{amsmath}   % for having text in math mode
\usepackage{extarrows} % for Row operations arrows
\usepackage{listings}
\lstset{
  frame=single,
  breaklines=true
}
\usepackage{hyperref}
  
%Following 2 lines were added to remove the blank page at the beginning
\usepackage{atbegshi}% http://ctan.org/pkg/atbegshi
\AtBeginDocument{\AtBeginShipoutNext{\AtBeginShipoutDiscard}}


%New macro definitions
\newcommand{\mydet}[1]{\ensuremath{\begin{vmatrix}#1\end{vmatrix}}}
\providecommand{\brak}[1]{\ensuremath{\left(#1\right)}}
\providecommand{\norm}[1]{\left\lVert#1\right\rVert}
\providecommand{\abs}[1]{\left\vert#1\right\vert}
\newcommand{\solution}{\noindent \textbf{Solution: }}
\newcommand{\myvec}[1]{\ensuremath{\begin{pmatrix}#1\end{pmatrix}}}
\let\vec\mathbf


\begin{document}

\begin{center}
\title{\textbf{Tangents and Normals}}
\date{\vspace{-5ex}} %Not to print date automatically
\maketitle
\end{center}
\setcounter{page}{1}

\section{12$^{th}$ Maths - Chapter 6}
This is Problem-2 from Exercise 6.3 
\begin{enumerate}
\item Find the slope of the tangent to the curve $y = \frac{x-1}{x-2}$, $x \neq 2$ at $x=10$.
\solution 
The given equation of the curve can be rearranged as
\begin{align}
	xy-x-2y+1 &= 0 \\
        \label{eq:Eq1}
	\implies \vec{x}^\top\myvec{0 & \frac{1}{2} \\ \frac{1}{2} & 0}\vec{x} + \myvec{-1 & -2}\vec{x}+1 &= 0 
\end{align}
The above equation can be equated to the generic equation of conic sections
\begin{align}
	\label{eq:Eq2}
	g\brak{\vec{x}} = \vec{x}^T\vec{V}\vec{x} + 2\vec{u}^T\vec{x} + f = 0 
\end{align}
Comparing coefficients of both equations \eqref{eq:Eq1} and \eqref{eq:Eq2} 
\begin{align}
	\vec{V} &= \myvec{ 0 & \frac{1}{2} \\ \frac{1}{2} & 0} \\
	\vec{u} &= -\myvec{\frac{1}{2} \\ 1} \\
	f &= 1 
\end{align}
Given the point of contact $\vec{q}$, the equation of a tangent to \eqref{eq:Eq2} is
\begin{align}
	\label{eq:Eq3}
	\brak{\vec{V}\vec{q} + \vec{u}}^\top \vec{x} + \vec{u}^\top\vec{q} + f = 0
\end{align}
		For the given point of contact with $\vec{q}_x=10$,
\begin{align}
	\vec{q}_y = \frac{10-1}{10-2} = \frac{9}{8} \\
	 \therefore \vec{q} = \myvec{10 \\ \frac{9}{8}}
\end{align}
\begin{align}
	& \eqref{eq:Eq3} \implies \brak{\myvec{ 0 & \frac{1}{2} \\ \frac{1}{2} & 0}\myvec{10\\\frac{9}{8}} - \myvec{\frac{1}{2} \\ 1}}^\top \vec{x} - \myvec{\frac{1}{2} & 1}\myvec{10 \\\frac{9}{8}} + 1 = 0  \\
	& \implies \brak{\myvec{ \frac{9}{16} \\ 5} - \myvec{\frac{1}{2} \\ 1}}^\top \vec{x} - \frac{41}{8} = 0 \\
	& \implies \myvec{\frac{1}{16} & 4}\vec{x} - \frac{41}{8} = 0 \\
	& \implies \vec{n} = \myvec{1 \\ 64}\\
	& \implies \vec{m} = \myvec{1 \\ \frac{-1}{64}}
\end{align}
The relevant diagram is shown in Figure \ref{fig:Fig1}
\begin{figure}[!h]
	\begin{center}
		\includegraphics[width=\columnwidth]{figs/problem2.pdf}
	\end{center}
\caption{}
\label{fig:Fig1}
\end{figure}
\end{enumerate}
\end{document}
