%\begin{enumerate}
\begin{enumerate}[label=\thesection.\arabic*.,ref=\thesection.\theenumi]
\item
Using the affine transformation in
\eqref{eq:conic_affine},
	the conic in     \eqref{eq:conic_quad_form} can be expressed in standard form 
	%(centre/vertex at the origin, major axis - $x$ axis)
	as
  \begin{align}
    %\begin{aligned}
    \label{eq:conic_simp_temp_nonparab}
	    \vec{y}^{\top}\brak{\frac{\vec{D}}{f_0}}\vec{y} &= 1   &  \abs{\vec{V}} &\ne 0
    \\
	    \vec{y}^{\top}\vec{D}\vec{y} &=  -\eta\vec{e}_1^{\top}\vec{y}   & \abs{\vec{V}} &= 0
    \label{eq:conic_simp_temp_parab}
    %\end{aligned}
    \end{align}
    where
  \begin{align}
      %\begin{split}
      \label{eq:f0}
	  f_0 &=\vec{u}^{\top}\vec{V}^{-1}\vec{u} -f \ne 0
	  \\
      \label{eq:eta}
       \eta &=2\vec{u}^{\top}\vec{p}_1
       \\
       \vec{e}_1 &=\myvec{1 \\ 0}
      \end{align}
     
    
\begin{proof}
  \label{app:parab}
	Using 
\eqref{eq:conic_affine}
%such that 
\eqref{eq:conic_quad_form} can be expressed as

%\item  
%Substituting \eqref{eq:conic_affine} in \eqref{eq:conic_quad_form}
\begin{align}
\brak{\vec{P}\vec{y}+\vec{c}}^{\top}\vec{V}\brak{\vec{P}\vec{y}+\vec{c}}+2\vec{u}^{\top}\brak{\vec{P}\vec{y}+\vec{c}}+ f
	= 0, 
\end{align}
yielding 
\begin{align}
\vec{y}^{\top}\vec{P}^{\top}\vec{V}\vec{P}\vec{y}+2\brak{\vec{V}\vec{c}+\vec{u}}^{\top}\vec{P}\vec{y}
+  \vec{c}^{\top}\vec{V}\vec{c} + 2\vec{u}^{\top}\vec{c} + f= 0
\label{eq:conic_simp_one}
\end{align}
%
From \eqref{eq:conic_simp_one} and \eqref{eq:conic_parmas_eig_def},
\begin{align}
\vec{y}^{\top}\vec{D}\vec{y}+2\brak{\vec{V}\vec{c}+\vec{u}}^{\top}\vec{P}\vec{y}
+  \vec{c}^{\top}\brak{\vec{V}\vec{c} + \vec{u}}+ \vec{u}^{\top}\vec{c} + f= 0
\label{eq:conic_simp}
\end{align}
When $\vec{V}^{-1}$ exists, choosing
\begin{align}
%\begin{split}
\vec{V}\vec{c}+\vec{u} &= \vec{0}, \quad \text{or}, \vec{c} = -\vec{V}^{-1}\vec{u},
\label{eq:conic_parmas_c_def}
\end{align}
%
%%From \eqref{eq:conic_parmas_k_def} and 
%%
and substituting \eqref{eq:conic_parmas_c_def}
in \eqref{eq:conic_simp}
yields \eqref{eq:conic_simp_temp_nonparab}. 
  %See Appendix \ref{app:parab}.
When $\abs{\vec{V}} = 0, \lambda_1 = 0$ and 
\begin{align}
\vec{V}\vec{p}_1 = 0, 
\vec{V}\vec{p}_2 = \lambda_2\vec{p}_2.
\label{eq:conic_parab_eig_prop} 
\end{align}
where $\vec{p}_1,\vec{p}_2$ are the eigenvectors of $\vec{V}$ such that  \eqref{eq:conic_parmas_eig_def}
%
\begin{align}
\vec{P} = \myvec{\vec{p}_1 & \vec{p}_2},
\label{eq:eig_matrix}
\end{align}
Substituting \eqref{eq:eig_matrix}
in \eqref{eq:conic_simp},
\begin{align}
	\vec{y}^{\top}\vec{D}\vec{y}+2\brak{\vec{c}^{\top}\vec{V}+\vec{u}^{\top}}\myvec{\vec{p}_1 & \vec{p}_2}\vec{y}
	+  \vec{c}^{\top}\brak{\vec{V}\vec{c} + \vec{u}}+ \vec{u}^{\top}\vec{c} + f&= 0
\\
\implies \vec{y}^{\top}\vec{D}\vec{y}
+2\myvec{\brak{\vec{c}^{\top}\vec{V}+\vec{u}^{\top}}\vec{p}_1  \brak{\vec{c}^{\top}\vec{V}+\vec{u}^{\top}}\vec{p}_2}\vec{y}
	+  \vec{c}^{\top}\brak{\vec{V}\vec{c} + \vec{u}}+ \vec{u}^{\top}\vec{c} + f&= 0
\\
\implies \vec{y}^{\top}\vec{D}\vec{y}
+2\myvec{\vec{u}^{\top}\vec{p}_1 & \brak{\lambda_2\vec{c}^{\top}+\vec{u}^{\top}}\vec{p}_2}\vec{y}
	+  \vec{c}^{\top}\brak{\vec{V}\vec{c} + \vec{u}}+ \vec{u}^{\top}\vec{c} + f&= 0
\end{align}
upon substituting from 
 \eqref{eq:conic_parab_eig_prop} yielding
\begin{align}
\lambda_2y_2^2+2\brak{\vec{u}^{\top}\vec{p}_1}y_1+  2y_2\brak{\lambda_2\vec{c}+\vec{u}}^{\top}\vec{p}_2
	+  \vec{c}^{\top}\brak{\vec{V}\vec{c} + \vec{u}}+ \vec{u}^{\top}\vec{c} + f= 0
\label{eq:conic_parab_foc_len_temp} 
\end{align}
%which is the equation of a parabola. 
Thus, \eqref{eq:conic_parab_foc_len_temp} 
can be expressed as \eqref{eq:conic_simp_temp_parab} by choosing
\begin{align}
%\label{eq:eta}
\eta = 2\vec{u}^{\top}\vec{p}_1
\end{align}
%Choosing 
%\begin{align}
%\vec{u} + \lambda_2\vec{c} = 0,
%\vec{c}^{\top}\brak{\vec{V}\vec{c} + \vec{u}}+ \vec{u}^{\top}\vec{c} + f = 0,
%\end{align}
% the above equation becomes
%\begin{align}
%y_2^2= -\frac{2\vec{u}^{\top}\vec{p}_1}{ \lambda_2} \brak{y_1
%+  \frac{\vec{u}^{\top}\vec{V}\vec{u} - 2\lambda_2\vec{u}^{\top}\vec{u} + f\lambda_2^2}{2\vec{u}^{\top}\vec{p}_1\lambda_2^2}}
%\\
%or \eta = 2\vec{u}^{\top}\vec{p}_1
%%\label{eq:conic_simp_parab_new}
%\end{align}
and $\vec{c}$ in \eqref{eq:conic_simp} such that
\begin{align}
\label{eq:conic_parab_one}
2\vec{P}^{\top}\brak{\vec{V}\vec{c}+\vec{u}} &= \eta\myvec{1\\0}
\\
\vec{c}^{\top}\brak{\vec{V}\vec{c} + \vec{u}}+ \vec{u}^{\top}\vec{c} + f&= 0
\label{eq:conic_parab_two}
\end{align}
%we obtain  \eqref{eq:conic_simp_temp_parab}.
$\because
\vec{P}^{\top}\vec{P} = \vec{I}$,
multiplying \eqref{eq:conic_parab_one} by $\vec{P}$ yields
\begin{align}
\label{eq:conic_parab_one_eig}
	\brak{\vec{V}\vec{c}+\vec{u}} &= \frac{\eta}{2}\vec{p}_1,
\end{align}
which, upon substituting in \eqref{eq:conic_parab_two}
results in 
\begin{align}
\frac{\eta}{2}\vec{c}^{\top}\vec{p}_1 + \vec{u}^{\top}\vec{c} + f&= 0
\label{eq:conic_parab_two_eig}
\end{align}
\eqref{eq:conic_parab_one_eig} and \eqref{eq:conic_parab_two_eig} can be clubbed together to obtain \eqref{eq:conic_parab_c}.
  \end{proof}
	  \item
		For the standard conic, 
				\begin{align}
					\label{eq:std-prm-P}
					\vec{P} &= \vec{I}
					\\
					\vec{u} &= 
				\begin{cases}
				0 & e \ne 1
       \\
				\frac{\eta}{2} \vec{e}_1 & e = 1
				\end{cases}
				\label{eq:std-prm-u}
				\\
				\lambda_1 &  
					\begin{cases}
						=0 & e = 1
						\\
						\ne 0 & e \ne 1
					\end{cases}
				\label{eq:std-prm-lam1}
				\end{align}
				where 
				\begin{align}
					\vec{I} = \myvec{\vec{e}_1 & \vec{e}_2}
				\end{align}
				is the identity matrix.
	  
    \item\leavevmode
		\begin{enumerate}
			\item The directrices for the  standard conic are given by 
				\begin{align}
					\label{eq:dx-ell-hyp}
					\vec{e}_1^{\top}\vec{y} &=  
					%\pm\sqrt{\abs{\frac{f_0\lambda_2}{\lambda_1\brak{\lambda_2-\lambda_1}}}} & e \ne 1
					\pm \frac{1}{e}\sqrt{\frac{\abs{f_0}}{\lambda_2\brak{1-e^2}}} & e \ne 1
					\\
					\vec{e}_1^{\top}\vec{y} &= \frac{\eta}{2\lambda_2} & e = 1
					\label{eq:dx-parab}
				\end{align}
    \item The foci of the standard ellipse and hyperbola are given by 
				\begin{align}
					\label{eq:F-ell-hyp-parab}
					\vec{F} 
=
					\begin{cases}
						\pm e\sqrt{\frac{\abs{f_0}}{\lambda_2\brak{1-e^2}}}\vec{e}_1 & e \ne 1
					%	\pm \sqrt{\abs{\frac{f_0}{\lambda_1}\brak{1 - \frac{\lambda_1}{\lambda_2}}}}\vec{e}_1 & e \ne 1
						\\
						 -\frac{\eta}{4\lambda_2}\vec{e}_1 & e = 1
					\end{cases}
				\end{align}
	
		\end{enumerate}
	%	where, without loss of generality, $f_0 < 0$ for the hyperbola.
    
	\begin{proof}%\leavevmode
  \label{app:foc-dir}
%  \input{appendix.tex}
		\begin{enumerate}
			\item For the standard hyperbola/ellipse in \eqref{eq:conic_simp_temp_nonparab}, from 
					\eqref{eq:std-prm-P},
\eqref{eq:conic_quad_form_nc}
and 
					\eqref{eq:std-prm-u},
				\begin{align}
\label{eq:n-ell-hyp}
					\vec{n} &= \sqrt{\frac{\lambda_2}{f_0}} \vec{e}_1 
					\\
					c &= 
					%\pm \frac{\sqrt{-\lambda_2\brak{e^2-1}\brak{\lambda_2 f_0}}}{\lambda_2e\brak{e^2-1}}
					\pm \frac{\sqrt{-\frac{\lambda_2}{f_0}\brak{e^2-1}\brak{\frac{\lambda_2}{ f_0}}}}{\frac{\lambda_2}{f_0}e\brak{e^2-1}}
					\\
					&=\pm \frac{1}{e\sqrt{1-e^2}}
%					\\
%					&=\pm\sqrt{\abs{\frac{f_0}{\brak{1 - \frac{\lambda_1}{\lambda_2}}\frac{\lambda_1}{\lambda_2}}}}
\label{eq:c-ell-hyp}
				\end{align}
				yielding 
					\eqref{eq:dx-ell-hyp} upon substituting from 
\eqref{eq:conic_quad_form_e} and simplifying.
For the standard parabola in \eqref{eq:conic_simp_temp_parab},  from 
					\eqref{eq:std-prm-P},
\eqref{eq:conic_quad_form_nc}
and 
					\eqref{eq:std-prm-u}, noting that $f = 0$,

				\begin{align}
\label{eq:n-parab}
					\vec{n} &= \sqrt{\lambda_2} \vec{e}_1 
					\\
					c &=
	\frac{\norm{\frac{\eta}{2} \vec{e}_1}^2   }{2\vec{\brak{\frac{\eta}{2}} \brak{\vec{e}_1}^{\top}\vec{n}}} 
\\
					\\
					&= \frac{\eta}{4\sqrt{\lambda_2}}
\label{eq:c-parab}
				\end{align}
				yielding 
					\eqref{eq:dx-parab}.

				\item 	For the standard ellipse/hyperbola, substituting from
\eqref{eq:c-ell-hyp},
\eqref{eq:n-ell-hyp},
\eqref{eq:std-prm-u}
and \eqref{eq:conic_quad_form_e}
in \eqref{eq:conic_quad_form_F},
				\begin{align}
					\vec{F} &= \pm \frac{\brak{\frac{1}{e\sqrt{1-e^2}}}\brak{e^2}\sqrt{\frac{\lambda_2}{f_0}}\vec{e}_1}{\frac{\lambda_2}{f_0}}
					%\pm\sqrt{\abs{\frac{f_0}{\brak{1 - \frac{\lambda_1}{\lambda_2}}\frac{\lambda_1}{\lambda_2}}}}
					%\brak{1 - \frac{\lambda_1}{\lambda_2}}\frac{\sqrt{\lambda_2}}{\lambda_2}\vec{e}_1
 			\end{align}
			yielding
					\eqref{eq:F-ell-hyp-parab}
					after simplification.
					For the standard parabola, substituting from 
\eqref{eq:c-parab},
\eqref{eq:n-parab},
\eqref{eq:std-prm-u}
and \eqref{eq:conic_quad_form_e}
in \eqref{eq:conic_quad_form_F},			
				\begin{align}
	\vec{F}  &= \frac{\brak{\frac{\eta}{4\sqrt{\lambda_2}}}\sqrt{\lambda_2}\vec{e}_1-\vec{\frac{\eta}{2} \vec{e}_1}}{\lambda_2}
\\
				\end{align}
				yielding 
					\eqref{eq:F-ell-hyp-parab} after simplification.

		\end{enumerate}
%		See Appendix \ref{app:foc-dir}.
	\end{proof}
	\end{enumerate}
