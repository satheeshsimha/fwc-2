\documentclass[12pt]{article}
\usepackage{graphicx}
\usepackage[none]{hyphenat}
\usepackage{graphicx}
\usepackage{listings}
\usepackage[english]{babel}
\usepackage{graphicx}
\usepackage{caption} 
\usepackage{booktabs}
\usepackage{array}
\usepackage{amssymb} % for \because
\usepackage{amsmath}   % for having text in math mode
\usepackage{extarrows} % for Row operations arrows
\usepackage{listings}
\lstset{
  frame=single,
  breaklines=true
}
\usepackage{hyperref}
  
%Following 2 lines were added to remove the blank page at the beginning
\usepackage{atbegshi}% http://ctan.org/pkg/atbegshi
\AtBeginDocument{\AtBeginShipoutNext{\AtBeginShipoutDiscard}}
\usepackage{gensymb}


%New macro definitions
\newcommand{\mydet}[1]{\ensuremath{\begin{vmatrix}#1\end{vmatrix}}}
\providecommand{\brak}[1]{\ensuremath{\left(#1\right)}}
\providecommand{\sbrak}[1]{\ensuremath{{}\left[#1\right]}}
\providecommand{\norm}[1]{\left\lVert#1\right\rVert}
\providecommand{\abs}[1]{\left\vert#1\right\vert}
\newcommand{\solution}{\noindent \textbf{Solution: }}
\newcommand{\myvec}[1]{\ensuremath{\begin{pmatrix}#1\end{pmatrix}}}
\let\vec\mathbf


\begin{document}

\begin{center}
\title{\textbf{Geometric Programming}}
\date{\vspace{-5ex}} %Not to print date automatically
\maketitle
\end{center}
\setcounter{page}{1}

\section{12$^{th}$ Maths - Chapter 6}
This is Problem-26 from Exercise 6.5 
\begin{enumerate}
\item Show that semi-vertical angle of right circular cone of given surface area and maximum volume is $sin^{-1}\brak{\frac{1}{3}}$.

\solution 
\item Let $r,h,l$ be the radius, height and slant height of the right circular cone respectively. Let $S$ be the given surface area and $V$ be the volume of the cone. We have 
\begin{align}
	l^2 &= r^2 + h^2 \\
	S &= \pi rl + \pi r^2 \\
	\implies l &= \frac{S-\pi r^2}{\pi r}\\
	V &= \frac{1}{3}\pi r^2h \\
	 &= \frac{1}{3}\pi r^2\sqrt{l^2-r^2} \\
	V^2 &= \frac{1}{9}\pi^2 r^4\brak{l^2-r^2} \\
	&= \frac{1}{9}\pi^2 r^4\brak{\brak{\frac{S-\pi r^2}{\pi r}}^2-r^2} \\
	&= \frac{1}{9}\pi^2 r^4\brak{\frac{\brak{S-\pi r^2}^2-\pi^2r^4 }{\pi^2 r^2}} \\
	&= \frac{1}{9} r^2\brak{\brak{S-\pi r^2}^2-\pi^2r^4 }
\end{align}
\begin{align}
	&= \frac{1}{9} r^2\brak{S^2- 2\pi S r^2 + \pi^2r^4 -\pi^2r^4 } \\
	&= \frac{1}{9} \brak{S^2r^2- 2\pi S r^4 } 
\end{align}
Differentiating wrt $r$,
\begin{align}
	2V \frac{dV}{dr} &= \frac{S^2}{9}2r - \frac{2\pi S}{9}4r^3 \\ 
	&= \frac{2rS}{9}\brak{ S- 4\pi r^2} 
\end{align}
For maximum volume, $\frac{dV}{dr} = 0$
\begin{align}
 	\implies  \frac{2rS}{9}\brak{ S- 4\pi r^2} &= 0 \\ 
	\implies r = 0 \text{ or } S - 4\pi r^2 = 0 
\end{align}
Since $r$ can't equal to 0,
\begin{align}
	\implies S - 4\pi r^2 &= 0 \\
	\implies S &= 4\pi r^2 \\
	\implies r^2 &= \frac{S}{4\pi} \\
	\implies r^2 &= \frac{\pi rl+\pi r^2}{4\pi} \\
	\implies 4\pi r^2 &= \pi rl+ \pi r^2 \\
        \implies 3\pi r^2 &= \pi rl \\
	\implies l &= 3r 
\end{align}
Let $\theta$ be the semi-vertical angle. Then,
\begin{align}
	sin\theta &= \frac{r}{l} \\
	sin\theta &= \frac{r}{3r} \\
	\implies \theta = sin^{-1}\frac{1}{3}
\end{align}
The relevant diagram is shown in Figure \ref{fig:Fig1}
\begin{figure}[!h]
	\begin{center}
		\includegraphics[width=\columnwidth]{figs/problem26.pdf}
	\end{center}
\caption{}
\label{fig:Fig1}
\end{figure}
\end{enumerate}
\end{document}
